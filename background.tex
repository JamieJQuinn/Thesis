\chapter{Background Physics}

\section{Introduction to solar physics}

\subsection{Layers of the sun}

The radius of stars is estimated as the radius at which the optical depth, a measure of the opacity of a plasma, takes the value $2/3$. In our own sun, this layer is found at a radius of approximately $R_{\odot} = 695,508$km and is known as the photosphere. Below this layer lie the three inner regions: the core, the radiative region and the convection zone. From the centre to around $0.2 R_{\odot}$ lies the engine of the sun, the core, undergoing fusion and producing the energy fuelling the [TODO rest of the sun]. Between the core and the convection zone (at $0.71R_{\odot}$) is the radiative zone, the region at which the density gradient is high enough that the plasma is stable to convective instabilities. Heat is transported from the core through the radiative zone mostly through the process of radiative transfer. Between the radiative zone and the photosphere lies the conduction zone, the region at which the density gradient is no longer able to stabilise the plasma against the effect of the negative temperature gradient and convection occurs. The convection within this region drives the solar dynamo and, thus, the creation of the solar magnetic field.

Above the photosphere lies the atmosphere of the sun, consisting of the photosphere, chromosphere, transition region and corona. The photosphere, usually referred to as the surface of the sun, has a temperature of approximately $6000$K and radiates most of the visible light emitted from the Sun. As a result the photosphere is the only part of the Sun usually visible by the human eye (the exception being during eclipses when both the chromosphere and corona can be seen). Just above the photosphere is the chromosphere, extending beyond the surface by $3000$--$5000$km. The temperature ranges between $6000$K at the photospheric boundary, a minimum of $3500$K internally, and $25,000$K approaching the transition region. This narrow layer of depth $100$km lies between the chromosphere and the Solar corona and is the point at which the nature of the physics in the Solar atmosphere changes rapidly with height. The primary difference is in the extreme temperature difference across the layer, typically ranging from $25,000$K on the chromospheric side, to over $1,000,000$K on the coronal side. TODO mention temperaature catastrophe? This coincides with a drop in density of the layer, from TODO to TODO. The other major difference is in the dominance of certain types of dynamics. While the dynamics of the chromosphere is dominated by the fluid dynamics of the plasma, and merely influenced by the magnetic field, the dynamics of the Solar corona is heavily dominated by the field, with the rarefied plasma being strongly affected by the magnetic field.

\subsection{Coronal heating problem}

refer to Browning, Hood, Klimchuck

\section{MHD Equations}

TODO copy from kink chapter

\section{Viscosity}

Physically, viscosity is the internal friction of a fluid arising due to interactions (typically collisions) between the particles making up the fluid. Within the context of MHD, viscosity provides two functions. The first is momentum transport, included in the momentum equation as the divergence of the viscous stress tensor $\ten{\sigma}$, written using Einstein notation for clarity,

\begin{equation}
  \label{eq:viscous_momenum_transport}
  (\nabla \cdot \ten{\sigma})_j = \nabla_i \sigma_{ij}.
\end{equation}

In three dimensions, the nine components of $\sigma$ quantify the flux (due to viscosity) of each component of momentum in each direction of motion. For example, the $\sigma_{xy}$ component gives the flow of $x$-momentum in the $y$-direction. Due to symmetry arising from viscosity naturally conserving angular momentum, the tensor is itself symmetric, so the component $\sigma_{xy}$ also quantifies the flow of $y$-momentum in the $x$-direction. The second function of viscosity is to convert kinetic into thermal energy through work done by local deformations. This is encoded in a term in the energy equation of the MHD equations, again expressed in Einstein notation,

\begin{equation}
  \label{eq:viscous_heat_generation}
  \ten{\sigma} : \nabla \vec{u} = \sigma_{ij} \frac{\partial u_i}{\partial x_j}.
\end{equation}

Beyond the physical requirement of conservation of angular momentum, it is assumed that Stokes hypothesis holds, that is bulk viscosity is zero and viscosity does not act under uniform compression or expansion of the fluid. This requires the viscous stress tensor to be trace-free,

\begin{equation}
  \label{eq:trace_free_tensor}
  \text{tr}(\ten{\sigma}) = 0.
\end{equation}

Many fluids encountered in daily life are Newtonian fluids, that is the viscous stress arising from any deformation of the fluid is directly proportional to the rate of strain of the deformation,

\begin{equation}
  \label{eq:isotropic_viscous_tensor}
  \ten{\sigma} = \nu \ten{W},
\end{equation}

where the traceless strain rate tensor $\ten{W}$ is written as

\begin{equation}
  \label{eq:rate_of_strain}
  \ten{W} = \nabla\vec{u} + (\nabla\vec{u})^T - \tfrac{2}{3}(\nabla \cdot \vec{u})\ten{I}.
\end{equation}

\subsection{Anisotropic Viscosity}

In a Newtonian fluid, the motion of a single particle travelling with a typical thermal velocity $v$ and colliding with other particles once in a typical collision time $\tau$ will appear as a number of broken, straight lines, each of approximate length $l = v\tau$, also known as the mean free path. These motions have no preferred direction, resulting in isotropic transfer of momentum. In contrast, in a plasma made up of charged particles with charge $e$ and mass $m$, threaded by a magnetic field of strength $B$, the particles follow helical paths of approximate radius $r = v/\omega$, where $\omega = eB/mc$ is the cyclotron frequency. After a time $\tau$, a typical particle will undergo a collision and its path will describe a new helix. The total resultant motion depends on the strength of the magnetic field. In the presence of a weak field, the radius of the helix may be much larger than the mean free path or, in terms of the cyclotron frequency, $\omega \tau \ll 1$. As a result, the path between collisions will be close to straight and the total path will resemble that of the motion without a magnetic field. In the presence of a strong field, $\omega \tau \gg 1$ and a typical particle will be able to wind around the field a number of times, travelling a distance $l$ along the field, before colliding. As a result the transport of momentum (i.e. viscosity) is strongly anisotropic: unaffected in the direction of the field, but strongly reduced in the transverse direction.

A characteristic coronal value of $\omega \tau$ can be calculated using expressions found in Braginskii~\cite{braginskiiTransportProcessesPlasma1965}. The collision time can be written in SI units as,

\begin{equation}
  \label{eq:collision_time}
  \tau = 0.82 \times 10^{6} \frac{T^{3/2}}{n},
\end{equation}

and the cyclotron frequency as,

\begin{equation}
  \label{eq:cyclotron_frequency}
  \omega = 0.8\times10^8 B,
\end{equation}

where the Coloumb algorithm has been taken to be 22, and the mass fraction, the ratio of ion to proton mass, $m_f = m_i/m_p$ has been taken to be a typical solar value of $1.2$. A solar active region could have typical temperatures of around $2\times 10^6$K, and number densities of $n = 3 \times 10^9$, giving $\tau = 0.77$s. The magnetic field can be as strong as $5\times 10^{-3}$T, giving $\omega = 4.1 \times 10^5 \text{s}^{-1}$, resulting in $\omega \tau = 3.16 \times 10^5$. This indicates viscosity in the solar corona is extremely anisotropic.

For educational purposes, I present a condensed review of Braginskii's qualitative derivation of the form of the anisotropic viscosity stress tensor in a magnetized plasma, and its associated transport parameters~\cite{braginskiiTransportProcessesPlasma1965}. Consider a plasma where, on average, each particle moves a distance $\Delta x$ in the collision time $\tau$ before colliding. After the collision the particle has equal probability of moving to the left and the right. Since we are concerned with the viscous diffusion of momentum, and not advection, we consider the case where the particle flux through some plane $x=x_0$ is zero (that is the number of particles moving through the plane from the left is equal to the number of particles from the right). This implies a uniform number density in the small layer around $x_0$. We do however consider a non-uniform velocity, say $u_y$, which changes slowly enough over the distance $\Delta x$ that we may write

\begin{equation}
  \label{eq:viscous_derivation_vy}
u_y (x) = u_y(x_0) + \frac{\partial u_y}{\partial x} \right|_{x=x_0} (x - x_0).
\end{equation}

Within some time, half the particles in the layer between $x_0 - \Delta x$ and $x_0$ will pass through the plane $x_0$, the other half moving in the opposite direction. The flux of $y$-momentum from the left is

\begin{equation}
  \label{eq:momentum_flux_left}
F_{+} = \frac{1}{2} \int^{x_0}{x_0 - \Delta x} \frac{1}{\tau} m n u_y(x) \text{dx} = \frac{mn}{2} \left[ u_y(x_0) - \frac{\partial u_y}{\partial x} \frac{\Delta x}{2} \right] \frac{\Delta x}{\tau},
\end{equation}

and the flux from the right $F_{-}$ can be calculated in a similar manner by considering the flux from the layer between $x_0$ and $x_0 + \Delta x$. The total flux $F = F_+ - F_-$ is

\begin{equation}
  \label{eq:total_momentum_flux}
  F = - \frac{nm(\Delta x)^2}{2\tau} \frac{\partial u_y}{\partial x},
\end{equation}

giving a qualitative estimate for the size of the $xy$-term of the viscous stress tensor, $\ten{\sigma}_{xy}$. 

\subsection{Alternative formulations of anisotropic viscosity}

The form of the viscous stress tensor in a magnetized plasma has been derived in a number of ways, to varying degrees of accuracy. The derivations typically use the methods of kinetic theory, taking moments of the Boltzmann-Maxwell equations, to arrive at approximations to the viscosity, among other types of molecular transport. A first approximation of the stress tensor can be found in the 1939 work of Chapman and Cowling~\cite{chapmanMathematicalTheoryNonuniform1970}. Their results show that the stress response to a rate of strain can be split into the responses to three types of motion: compression or dilation along the field, deformations in the plane transverse to the field, and deformations in the plane including the field. The latter two responses can each be further split into two parts, giving five distinct contributions to the stress tensor in total. This natural splitting of the tensor in response to various kinds of strain is discussed in depth by Kaufman~\cite{kaufmanPlasmaViscosityMagnetic1960}. In the same article, Kaufman presents both an illustrative description of the drift and perpendicular components of the tensor, and a derivation of the full tensor from a more simplified Boltzmann equation than is used by Chapman and Cowling. The tensor derived by Braginskii~\cite{braginskiiTransportProcessesPlasma1965} is perhaps the most well known and includes accurate approximations to the five viscous transport parameters. The parallel component of the stress tensor has been derived without use of kinetic theory by Hollweg~\cite{hollwegViscosityMagnetizedPlasma1985}, showing the viscous response to parallel motions to be a result of collisions repartitioning anisotropies in the thermal pressure. 

\section{Magnitude of transport parameters}

\subsection{Viscosity}

Hollweg gives $\eta_0 = 10^{-17} T^{5/2}\ \text{kg m}^{-1} \text{s}^{-1}$ for a Coulomb logarithm of 22. For temperatures of $T=10^4$ (the initial condition in kinks) to $T=10^8$ (the approximate final temperature in the kink instability) $\eta_0$ ranges between $\eta_0 = 10^{-7}$ and $\eta_0 = 10^3$. For the normalisation used in the kink instabilties; $B_0 = 0.005\ \text{T}$, $L_0 = 10^6\ \text{m}$ and $\rho_0 = 1.67 \times 10^{-12}$, the Alfv\'en velocity is approximately $3.45 \times 10^6 \text{ms}^{-1}$ and hence the range of the Reynolds number is $\text{Re} = \frac{V_A L_0 \rho_0}{\eta_0} = 5.67 \times 10{-3}$--$5.67 \times 10^{7}$. For a more typical active region temperature of $T=10^6 \text{K}$, the Reynolds number is $5.67 \times 10^2$. 

\subsection{Resistivity}

Hollweg's expression for the resistivity is $D = 2 \times 10^{9} T^{-3/2} \text{m}^2 \text{s}^{-1}$ [Does this correspond to Spitzer resistivity?]. This assumes the electron temperature is the same as the ion temperature. A temperature of $T=10^6\ \text{K}$ results in $D = 2 \text{m}^2 \text{s}^{-1}$. This corresponds to a Lundquist number of $S = \frac{V_A L_0}{D} = 1.73 \times 10^{12}$. Including the effects of electron pitch-angle scattering or current-driven instabilities can increase the effective resistivity. Exactly how much the resistivity is generally increased by is debated and ultimately depends on the local plasma conditions. Estimates range over Lundquist numbers of $10^{4}$--$10^{8}$ [Add reference to these estimates from Ian Craig's paper].

What should be noted is the magnetic Prandtl number, the ratio of Lundquist to Reynolds number, could be as large as $\text{Pr}_m = 10^2$ at a temperature of $T=10^6\ \text{K}$, taking into account an extremely large increase in effective resistivity. Realistically, assuming a more moderate increase in effective resistivity, the magnetic Prandtl number could be as high as $10^{6}$.

[Question - Should I stick to speaking exclusively about Reynolds \& Lundquist, or instead viscosity and resistivity?]
