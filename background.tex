\chapter{Background Physics}

\section{Introduction to solar physics}

- Coronal heating problem
- Layers of sun
- Plasma physics

\section{MHD Equations}

TODO copy from kink chapter

\section{Viscosity}

Basics

\section{Magnitude of transport parameters}

\subsection{Viscosity}

Hollweg gives $\eta_0 = 10^{-17} T^{5/2}\ \text{kg m}^{-1} \text{s}^{-1}$ for a Coulomb logarithm of 22. For temperatures of $T=10^4$ (the initial condition in kinks) to $T=10^8$ (the approximate final temperature in the kink instability) $\eta_0$ ranges between $\eta_0 = 10^{-7}$ and $\eta_0 = 10^3$. For the normalisation used in the kink instabilties; $B_0 = 0.005\ \text{T}$, $L_0 = 10^6\ \text{m}$ and $\rho_0 = 1.67 \times 10^{-12}$, the Alfv\'en velocity is approximately $3.45 \times 10^6 \text{ms}^{-1}$ and hence the range of the Reynolds number is $\text{Re} = \frac{V_A L_0 \rho_0}{\eta_0} = 5.67 \times 10{-3}$--$5.67 \times 10^{7}$. For a more typical active region temperature of $T=10^6 \text{K}$, the Reynolds number is $5.67 \times 10^2$. 

\subsection{Resistivity}

Hollweg's expression for the resistivity is $D = 2 \times 10^{9} T^{-3/2} \text{m}^2 \text{s}^{-1}$ [Does this correspond to Spitzer resistivity?]. This assumes the electron temperature is the same as the ion temperature. A temperature of $T=10^6\ \text{K}$ results in $D = 2 \text{m}^2 \text{s}^{-1}$. This corresponds to a Lundquist number of $S = \frac{V_A L_0}{D} = 1.73 \times 10^{12}$. Including the effects of electron pitch-angle scattering or current-driven instabilities can increase the effective resistivity. Exactly how much the resistivity is generally increased by is debated and ultimately depends on the local plasma conditions. Estimates range over Lundquist numbers of $10^{4}$--$10^{8}$ [Add reference to these estimates from Ian Craig's paper].

What should be noted is the magnetic Prandtl number, the ratio of Lundquist to Reynolds number, could be as large as $\text{Pr}_m = 10^2$ at a temperature of $T=10^6\ \text{K}$, taking into account an extremely large increase in effective resistivity. Realistically, assuming a more moderate increase in effective resistivity, the magnetic Prandtl number could be as high as $10^{6}$.

[Question - Should I stick to speaking exclusively about Reynolds \& Lundquist, or instead viscosity and resistivity?]
