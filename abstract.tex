\chapter{Abstract}

Current understanding of the solar corona is plagued by what is known as the coronal heating problem--it is unknown how the solar atmosphere maintains a temperature of several orders of magnitude greater than that of the solar surface. Viscosity provides one mechanism by which  heat is generated, through the dissipation of kinetic energy. Although Newtonian viscosity is a common feature of many coronal simulations, the proper form of viscosity in a highly magnetised plasma is anisotropic and strongly coupled to the local magnetic field. This thesis investigates the differences between Newtonian and a novel model of anisotropic viscosity, the switching model, when applied to a simulations of the kink instability in a coronal loop, a slowly stressed magnetic null point, and the Kelvin-Helmholtz instability in the fan plane of a null point. Generally it's found that Newtonian viscosity overestimates the viscous heating by up to two orders of magnitude and can suppress the growth of instabilities and current sheets, leading to notably diminished Ohmic heating and reconnection rates.
