\chapter{Abstract}

The kink instability of magnetohydrodynamics is believed to be fundamental to many aspects of the dynamic activity of the solar atmosphere, such as the initiation of flares and the heating of the solar corona. In this work, we investigate the importance of viscosity on the kink instability. In particular, we focus on two forms of viscosity; isotropic viscosity (independent of the magnetic field) and anisotropic viscosity (with a preferred direction following the magnetic field). Through the detailed analysis of magnetohydrodynamic simulations of the kink instability with both types of viscosity, we show that the form of viscosity has a significant effect on the nonlinear dynamics of the instability. The different viscosities allow for different flow and current structures to develop, thus affecting the behaviour of magnetic relaxation, the formation of secondary instabilities and the Ohmic and viscous heating produced. Our results have important consequences for the interpretation of solar observations of the kink instability.
