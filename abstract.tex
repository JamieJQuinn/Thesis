\chapter{Abstract}

The dissipation of kinetic energy through viscosity provides one mechanism by which the solar atmosphere may be heated. Although isotropic, Newtonian viscosity is a common feature of many coronal simulations, the proper form of viscosity in a highly magnetised plasma is anisotropic and strongly coupled to the local magnetic field. This thesis investigates the differences between isotropic viscosity and a novel model of anisotropic viscosity, the switching model, when applied to simulations of the kink and fluting instabilities in a coronal loop, a slowly stressed magnetic null point, and the Kelvin-Helmholtz instability in the fan plane of a null point. The choice of viscosity model strongly affects the stability and evolution of the studied instabilities, and the heating generated in their development. The use of anisotropic viscosity generally diminishes viscous heating, enhances Ohmic heating, produces small scales in flow and current structures, and results in more energetic instabilities.
