\chapter{Application to the kink instability II: Initially straight}

\label{chp:kink_instability_straight}

\graphicspath{{images/kink_instability_straight/}}

\section{Introduction}

In chapter~\ref{chp:kink_instability} the dynamics in a magnetic flux rope which is already linearly unstable to the helical kink instability were investigated. An alternative way to excite the kink instability is to start with an initially straight field and apply twisting motions at the boundaries to create a flux rope which eventually becomes unstable to the kink instability. In addition, a fluting instability is found to be excited prior to the development of the kink instability. To our knowledge, this is the first time a fluting instability in coronal conditions has been studied through 3D numerical simulations. 

\todo{what is fluting instability}

\todo{who has studied fluting instability before?}

\section{Model setup}

To investigate a simple, null-free topology, we start with a straight, uniform field
\begin{equation}
\vec{B} = (0, 0, 1)^{\text{T}},
\end{equation}
in a cube of dimension $[-2,2]^3$. The velocity is set everywhere to $\vec{u} = \vec{0}$, the density to $\rho = 1$, the internal energy to $\varepsilon = 8.67\times 10^{-4}$ (corresponding to a temperature of $10^6$K). At the boundaries the magnetic field, density, and energy are fixed to their initial values and the velocity zero except at the upper and lower boundary where the twisting driver, described below, is prescribed. The spatial derivatives of these variables are also set to zero at the boundaries. The resolution is $512$ grid points per dimension, comparable to the highest resolution kink instability studies of~\cite{hoodCoronalHeatingMagnetic2009} or medium resolution studies of~\cite{barefordShockHeatingNumerical2015}.

\begin{figure}[t]
  \centering
  \begin{subfigure}{.49\textwidth}
  \centering
  \includegraphics[width=1.0\linewidth]{u_r.pdf}
  \caption{Radial dependence of driver}
  \label{fig:kink_radial_driver}
  \end{subfigure}
  \begin{subfigure}{.49\textwidth}
  \centering
  \includegraphics[width=1.0\linewidth]{u_t.pdf}
  \caption{Acceleration of driver}
  \label{fig:kink_driver_accel}
  \end{subfigure}
  
  \caption{Driver velocity radial profile $u(r)$ and acceleration profile $u(t)$.}
  \label{fig:kink_driver}
\end{figure}

\todo{field lines initially and after some twist}

The flux rope is formed by prescribing a slowly accelerating, rotating flow at the upper $z$-boundary as
\begin{equation}
  \label{eq:null_twisting_profile}
  \vec{u} = u_0 u_r(r) u_t(t) (-y, x),
\end{equation}
where $u_r(r)$ describes the radial profile of the twisting motion in terms of the radius $r^2 = x^2 + y^2$,
\begin{equation}
  \label{eq:radial_twisting_function}
  u_r(r) = 2r(1 + \tanh(1 - r_d r^2)),
\end{equation}
where $r_d$ controls the radial extent of the driver, and $u_t(t)$ describes the imposed acceleration of the twisting motion,
\begin{equation}
  \label{eq:ramping_up_function}
  u_t(t) = \tanh^2(t/t_r),
\end{equation}
where the parameter $t_r$ controls the time taken to reach the final driver velocity $u_0$. The functions $u(r)$ and $u(t)$ are plotted in figure~\ref{fig:kink_driver}. At the lower boundary, the flow is in the opposite direction. This driver, the same as is used in the chapter~\ref{chp:null_point_khi}, allows the driver to be accelerated slowly enough that disruptive shocks and fast waves are not generated. The driver does still generate Alfv\'en waves. The driver parameters used in this chapter are $u_0 = 0.45$, $r_d = 5$ and $t_r = 1$.

This configuration produces a $z$-directed tube of twisted magnetic field that eventually becomes unstable to the helical kink instability and erupts. Although the setup is similar to that of a twisted coronal loop, including the resulting kink instability, it is not intended to be a realistic simulation of an erupting loop. Instead, the intention here is to investigate the differences in the viscosity models in two different topologies under similar stressed conditions.

\section{Methods}

\subsection{Numerical setup}

\todo{numerical setup}

Switching parameter = 150, does this mean it's aniso all the time??

\subsection{Analysis tools}

\subsubsection{Reconnection rate}

The reconnection rate is calculated using the same method employed in chapter~\ref{chp:kink_instability}. In summary, the reconnection rate local to a given magnetic field line is calculated as the local parallel electric field (that is, parallel to the magnetic field) integrated along the field line. By choosing a grid of starting points and integrating along each associated field line, we construct an image of reconnection rates projected onto the grid of field line seed points. This is used to explore the spatial distribution of reconnection. By taking the maximum value across all seed points, we recover the accepted measure of reconnection rate, the maximum integrated value~\cite{galsgaardSteadyStateReconnection2011,priestNatureThreedimensionalMagnetic2003,schindlerGeneralMagneticReconnection1988}.

\section{Results}

\todo{plot KE}

The initial reaction to the moving boundaries is two torsional Alfv\'en waves which travel along the tube from the upper and lower boundaries to the opposite boundary. The interaction between these waves produces an oscillating pattern in the kinetic energy with a period of approximately $4$, equal to the time taken for a wave to travel the entire length of the domain (figure~TODO).

\todo{plot field lines with contours of energy to show instabilities}

The field continues to be twisted, injecting energy into the magnetic field, until the tube reaches a critical point and an instability is excited. Due to our choice of parameters, we see two different instabilities being excited. The common $m=1$ helical kink mode, seen in figure~\ref{fig:kink_isosurface}, is excited when the resistivity is set to the smaller value of $10^{-4}$, regardless of the choice of the viscosity parameter. The other mode we find is perhaps a resistive kink mode with $m>1$ as has been found in cylindrical tokamaks~\cite{furthTearingModeCylindrical1973a}. This mode is excited when the resistivity is set to the higher value of $10^{-3}$, again regardless of the viscosity, and can be seen in figure~\ref{fig:other_instability_isosurface}. Though of interest, an investigation into the onset and stability of this unknown mode is not the focus of this chapter.

\todo{plot azimuthal velocity showing patter in switching and none in isotropic}

Without viscous damping, a pattern in the azimuthal velocity forms at the centre TODO finish

\todo{plot lorentz + pressure forces comparing kink instability and fluting}

\section{Discussion}

\section{Conclusion}
