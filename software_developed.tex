\chapter{Software and reproduction of results}

This thesis has involved the development of a number of tools used to analyse the outputs from Lare3d, mostly written in Python. For the purpose of proper reproducibility, this appendix details the theory behind the field line integrator, the precise software versions and parameters used to run the simulations and analyses, and the locations of all relevant code and data. Due to the size of the output files from Lare3d (the total amount of data generated in the entire thesis is approximately 10 TB) only the code and relevant parameters are published. These should provide enough information to reproduce the simulation results, however if a required piece is missing, I encourage the reader to contact me.

\section{Reproduction of results}

With the exception of a field line integrator used in chapter~\ref{chp:kink_instability}, all analysis code is packaged alongside the latex files used to generate this thesis and can be found in the Github repository at \url{https://github.com/jamiejquinn/thesis} and is also stored in Zenodo~\cite{jamie_j_quinn_2020_4263146}. The theory behind each piece of analysis is described in the relevant chapters and instructions for rerunning any analysis can be found in the README of the thesis repository. The field line integrator used in chapter~\ref{chp:kink_instability} can be found at~\cite{jamie_j_quinn_2019_3560249} and is distinct from the alternative field line integrator used in other chapters which is described below and can be found alongside the other analysis tools in the thesis repository.

The code which implements only the anisotropic viscosity module can be found at~\cite{keith_bennett_2020_4155546} and should be simple to merge into another version of Lare3d for future research. To facilitate reproduction of the simulation data presented in this thesis, the code used in each chapter is individually packaged in different branches of the repository found at \url{https://github.com/jamiejquinn/Lare3d} and are also stored in Zenodo for chapter~\ref{chp:switching_model} at~\cite{keith_bennett_2020_4155661}, chapter at~\ref{chp:kink_instability} at~\cite{keith_bennett_2020_4155670}, chapter~\ref{chp:kink_instability_straight} at~\cite{keith_bennett_2020_4155625}, and chapter~\ref{chp:null_point_khi} at~\cite{keith_bennett_2020_4155646}. These versions of Lare3d include initial conditions, boundary conditions and basic running parameters. The specific parameters used in each individual simulation can be found in the methods sections of the corresponding chapters. The parameters were inputted to the simulations using the tools found in the \verb|run_scripts| folder of the thesis repository. These can be used to quickly generate multiple simulations suitable for a parameter study.

All simulations were performed on a single, multi-core machine with $40$ cores and $192$ GB of RAM, although this amount of RAM is much higher than was required; a conservative estimate of the memory used in the largest simulations is around $64$ GB. Most simulations completed in under $2$ days, although the longest running simulations (the highest-resolution cases in chapter~\ref{chp:null_point_khi}) completed in around $2$ weeks. 

\section{Field line integrator}

As detailed in section~\ref{sec:kink_methods_analysis}, the reconnection rate local to a single field line is given by the electric field parallel to the magnetic field, integrated along the field line. The global reconnection rate for a given region of magnetic diffusion is the maximum value of the local reconnection rate over all field lines threading the region. In chapter~\ref{chp:kink_instability} this was calculated using the visualisation tool Mayavi (more details are found in section~\ref{sec:kink_methods_analysis}) while in all other chapters, a field line integrator was developed specifically for the calculation of reconnection rate and is detailed here.

Magnetic field lines lie tangential to the local magnetic field at every point $\vec{x}(s)$ along the line,
\begin{equation}
  \label{eq:field_line_equation}
  \frac{d\vec{x}(s)}{ds} = \vec{b}(\vec{x}(s)),
\end{equation}
where $s$ is a variable which tracks along a single field line and $\vec{b}$ is the unit vector in the direction of $\vec{B}$. This equation is discretised using a second-order Runge-Kutta scheme to iteratively calculate the discrete positions $\vec{x}_i$ along a field line passing through some seed position $\vec{x}_0$,
\begin{align}
  \label{eq:field_line_calculation}
  \vec{x}_{i+1} &= \vec{x}_i + h\vec{b}(\vec{x}'_i),\\
  \vec{x}'_i &= \vec{x}_i + \tfrac{h}{2}\vec{b}(\vec{x}_i)
\end{align}
where $h$ is a small step size. Since $\vec{b}$ is discretised, the value at an arbitrary location $\vec{x}_i$ is calculated using a linear approximation. The integration of a scalar variable $y$ is carried out along a field line given by a sequence of $N$ locations $\vec{x}_i$ using the midpoint rule,
\begin{equation}
  \label{eq:midpoint_rule}
  Y = \sum_{i=1}^{i=N} \frac{(y(\vec{x}_{i-1}) + y(\vec{x}_{i}))}{2},
\end{equation}
where $Y$ is the result of the integration. In practice, $N$ is not specified and the discretised field line contains the required number of points to thread from its seed location to the boundary of the domain.

While the linear interpolation, second-order Runge-Kutta and midpoint rule are all low order methods, testing higher-order methods showed little change in results but dramatically increased the runtime of the analysis. The lower-order methods used offer an acceptable compromise between speed and accuracy. The above algorithm is implemented in Python and can be found in \verb|code/shared/field_line_integrator.py| with examples of use in \verb|code/null_point_khi/field_line_integrator.Rmd|. The integration of multiple field lines is an embarrassingly parallel problem and is parallelised in a straight-forward manner using a pool of threads supplied by the \verb|Pool| feature of the Python library \verb|multiprocessing|. Although the integrator is solely used to integrate the parallel electric field along magnetic field lines in this thesis, the tool can be easily applied to arbitrary vector and scalar fields.
