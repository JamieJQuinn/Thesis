\chapter{Introduction}

\begin{figure}
\centering
\includegraphics[scale=0.125]{GlaLogo.pdf}
\caption{This is the university logo, used as an example because it is needed for the front page.}
\label{logo figure}
\end{figure}

Here starts the thesis with some text to check that things are working. Let's refer to figure \ref{logo figure} to check that the captions and labels work. I shall also check that \textsf{sanserif} and \texttt{typewriter} work. And here is equation (\ref{inductor iv}) too,
\begin{equation}
v = L \frac{di}{dt} .
\label{inductor iv}
\end{equation}
A report is a formal document and should be written in appropriate language. Numerous books offer advice on writing reports and a selection is listed in the references at the end. Here are a few tips.
\begin{itemize}
\item
Reports should be written in correct English. Break text into paragraphs, keep sentences to a reasonable length and insert appropriate punctuation. Use a spell-checker and a grammar-checker if desired but neither is a substitute for careful reading.
\item
A report is not a story.
Write `The voltage was measured' rather than `I measured the voltage'. This document contains instructions and therefore uses a different style.
\item
Define all abbreviations when they are first used: `The accelerometer uses a serial peripheral interface (SPI)'. Provide a list of abbreviations if you use a large number of them.
\item
Don't write material that you don't understand. It will be obvious to the reader.
\end{itemize}
The quality of English is assessed as part of the report. Foreign students may feel this to be a burden but part of their education in this country is to learn to work effectively in an English-speaking environment.


\section{Precision}

An engineering report must be precise. This applies both to the language and to numerical values. For example, the words \emph{precision} and \emph{accuracy} are often used interchangably in non-technical discussion but the distinction between them is vital in engineering. Vague, waffly text is a major weakness of many students' technical reports (and examination answers).

\subsection{A subsection}

Figures (diagrams, photographs etc) and tables must have informative captions and be numbered. Axes of graphs should have scales, titles and units, otherwise the plot is meaningless. Multiple curves must be labelled, either directly or with a caption. Use dotted or dashed lines as well as colour for clarity; remember that the reader might be colour-blind or have only a black-and-white printout. All text must be legible, roughly the same size as the main text. Be warned that plots from Excel or Matlab need extensive editing to bring them up to an acceptable standard. Experimental traces can be captured on most modern test equipment and can make good illustrations.
