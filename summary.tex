\chapter{Final discussion}

\label{chp:summary}

As a whole, this thesis presents a detailed investigation of a new model of anisotropic viscosity, the switching model, and its use in a number of important coronal applications including the helical kink and fluting instabilities in a twisted magnetic flux rope and the Kelvin-Helmholtz instability in the fan plane of a twisted null point. The switching model captures the main physics of the classical Braginskii model (when applied to the solar corona) where momentum transport is isotropic in the vicinity of a null point and is otherwise (predominantly) parallel to the magnetic field. The switching model additionally exposes control over the effective size of the isotropic regions around magnetic null points. This final chapter presents a holistic discussion of general findings and motivates future research. For detailed discussions of the results of individual applications, see the discussion sections of the relevant chapters.

The choice of viscosity model in simulations of twisted magnetic flux ropes and stressed magnetic null points strongly affects the stability of the fluting, kink and Kelvin-Helmholtz instabilities, their linear and nonlinear evolution, and the viscous and Ohmic heating generated throughout their development. In general, isotropic viscosity has a pronounced damping effect on the growth of each instability, to the point of complete suppression in some cases. In the dynamically twisted flux rope, the growth of the fluting instability is suppressed to the extent it is completely disrupted by the simultaneously growing kink instability. In the stressed null point of chapter~\ref{chp:null_point_khi}, isotropic viscosity is so efficient at damping the KHI that only a single case is found where the instability notably breaks up the associated current-vortex sheet. Surprisingly, it is this efficient damping which allows the formation of a secondary instability after the onset of the kink instability in chapter~\ref{chp:kink_instability}, behaviour that is not observed when anisotropic viscosity is used.

In terms of heating, isotropic viscosity is generally found to overestimate viscous heating by up to two orders of magnitude, compared to the switching model. Despite this, the smaller length scales permitted by anisotropic viscosity tend to generate stronger current sheets and enhance Ohmic heating. In the applications studied here this results in an overall greater increase in total heating when using anisotropic viscosity. Although it is generally true that isotropic viscous heating outperforms anisotropic by several orders of magnitude, where the KHI occurs in the null point (in chapter~\ref{chp:null_point_khi}) the resultant flows are such that anisotropic viscous heating is massively enhanced and becomes comparable to isotropic heating. Hence, the overall effect of anisotropic viscosity on viscous heating in a given coronal application is fundamentally non-trivial and cannot be easily understood without investigation of relevant nonlinear phenomena. However, the general enhancement of heating by anisotropic viscosity in the applications studied here is encouraging in the context of coronal heating, and suggests that simulations using isotropic viscosity may generally be underestimating the degree of total heating.

While the specific estimates of Ohmic heating presented here rely on unrealistically large values for the resistivity (due to computational limits) the results of the various parameter studies show that an order of magnitude difference in the value of resistivity does not correspond to an order of magnitude difference in Ohmic heating, at least for kink-unstable flux ropes and stressed magnetic null points. If a realistic estimate of the resistivity is taken to be around four to eight orders of magnitude smaller than those studied here, these findings indicate that the predictions of Ohmic heating found in the results presented here may only be a few orders of magnitude greater than realistic levels. Contrastingly, since the value of viscosity used here is close to a realistic coronal value, the predictions of viscous heating generated by the anisotropic model give an estimate of the degree of true viscous heating in the solar corona. For the specific parameters studied here, Ohmic heating contributes the bulk of the total heating, outperforming viscous heating by several orders of magnitude, however this relationship may not hold for lower values of resistivity. 

As well as enhancing Ohmic heating, the stronger current sheets generated using anisotropic viscosity permit more efficient reconnection, generally enhancing the reconnection rate. This allows for faster magnetic relaxation and, in the case of the disruption of the secondary instability during the nonlinear development of the kink instability in chapter~\ref{chp:kink_instability}, more energetic reconnection-driven flows.

Generally, anisotropic viscosity allows greater release of kinetic energy and permits faster flow structures at smaller scales. In the nonlinear development of the kink instability, these faster flows result in the previously mentioned disruption of the secondary instability which is only present when isotropic viscosity is used and the flow remains relatively laminar. In simulations where the fluting instability occurs, the mixing caused by the nonlinear development of the instability appears chaotic and over a range of length scales. These results prompt a question for future research; are flows associated with anisotropic viscosity generally turbulent in nature? 

Two phenomena have been observed which were unexpected: the fluting instability in a dynamically twisted flux rope, and the spontaneous collapse of a twisted null point. Both the fluting instability and the specific mechanism causing the collapse of the null point are pressure driven and rely on spatial gradients in Ohmic heating to generate the required pressure gradients. It is unclear if the use of lower, more realistic values for the resistivity would provide the required Ohmic heating to generate these pressure gradients. Both phenomena are affected by the choice of viscosity model but do not rely specifically on either model; the fluting instability is damped by isotropic viscosity in some cases, while the null collapse occurs sooner when anisotropic viscosity is employed. The results presented here are a proof-of-concept, that the fluting instability can be dynamically excited in a coronal loop model, and that a null point can spontaneously collapse under the action of a torsional driver. Further investigation of either would be valuable avenues of future research.

The models of anisotropic viscosity employed throughout this thesis have mainly been the switching model with either the von Mises switching function or the parallel Braginskii switching function, both of which neglect the drift and perpendicular terms in Braginskii's full viscous stress tensor on the basis that the relative strengths of the associated transport parameters render these terms negligible. However, a finding general to all numerical experiments performed here has been the consistently smaller length scales and faster flows generated when using anisotropic viscosity, resulting in greater velocity shears. Are these shears great enough to result in notable perpendicular viscosity? Although the drift terms do not participate in heating, how might they affect the dynamics in a strong shear layer such as that studied in chapter~\ref{chp:null_point_khi}? These remain open questions to answer in future research.

However, without extreme velocity shears the perpendicular and drift components of the Braginskii tensor remain secondary in importance to the parallel and isotropic components. Hence, the switching model provides a valuable numerical tool in the simulation of coronal instabilities. For the model to be used in future research, chapter~\ref{chp:development_of_switching_model} should be consulted to understand the advantages of each switching function. The parallel switching function used in chapter~\ref{chp:null_point_khi} appears to capture the physics in a way closest to Braginskii's original model but at the expense of computational efficiency.

This thesis presents an initial, important foray into the application of anisotropic models of viscosity to nonlinear solar instabilities. The switching model provides a useful numerical tool to investigate the balance of isotropic and parallel viscosity in the solar corona. Along with the study of the KHI and kink instabilities, two novel phenomena, the nonlinear fluting instability in a coronal loop and the spontaneous collapse of a null point, are presented and warrant further investigation.
