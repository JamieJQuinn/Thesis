\chapter{Final thoughts}

\label{chp:summary}


As a whole, this thesis details a new model of anisotropic viscosity, the switching model, and its use in a number of important coronal applications: to the helical kink and fluting instabilities in a twisted magnetic flux rope, and to the Kelvin-Helmholtz instability in the fan plane of a twisted null point and eventual collapse of the null. The switching model captures the main physics of the classical Braginskii model (when applied to the solar corona) where momentum transport is isotropic in the vicinity of a null point and is otherwise parallel to the magnetic field. The switching model additionally exposes control over the effective size of the isotropic regions around magnetic null points. This final chapter presents a holistic discussion of general findings and motivates future research. For detailed discussions of the results of individual applications, see the discussion sections of the relevant chapters.

The choice of viscosity model in simulations of twisted magnetic flux ropes and stressed magnetic null points strongly affects the stability of the fluting, kink and Kelvin-Helmholtz instabilities, their linear and nonlinear evolution, and the viscous and Ohmic heating generated throughout their development. In general, isotropic viscosity has a damping effect on the growth of each instability, to the point of complete suppression in some cases. In the dynamically twisted flux rope, the growth of the fluting instability is damped to the extent it is completely disrupted by the simultaneously growing kink instability. In the stressed null point of chapter~\ref{chp:null_point_khi}, isotropic viscosity is so efficient at damping the KHI that only a single case is found where the instability notably breaks up the current-vortex sheet. Surprisingly, it is this efficient damping which allows the formation of a secondary instability after the onset of the kink instability in chapter~\ref{chp:kink_instability} which is not found when anisotropic viscosity is used.

In terms of heating, isotropic viscosity is generally found to overestimate viscous heating by up to two orders of magnitude, compared to the switching model. Despite this, the smaller length scales permitted by anisotropic viscosity tend to generate stronger current sheets and enhance Ohmic heating. In the applications studied here this results in an overall greater increase in total heating when using anisotropic viscosity. Although it is generally true that isotropic viscous heating outperforms anisotropic by several orders of magnitude, where the KHI occurs in the null point (in chapter~\ref{chp:null_point_khi}) the resultant flows are such that anisotropic viscous heating is massively enhanced and becomes comparable to isotropic viscous heating. Hence, the effect of anisotropic viscosity on viscous heating in a given coronal application is fundamentally non-trivial and cannot be easily understood without investigation of relevant nonlinear phenomena.

While the specific estimates of Ohmic heating presented here rely on unrealistically large values for the resistivity (due to computational limits) the results of the various parameter studies show that an order of magnitude difference in the value of resistivity does not correspond to an order of magnitude difference in Ohmic heating, at least for kink-unstable flux ropes and stressed magnetic null points. If a realistic estimate of the resistivity is taken to be around four to eight orders of magnitude smaller than those studied here, these findings indicate that the predictions of Ohmic heating found in the presented simulation results may only be a few orders of magnitude greater than realistic levels. Contrastingly, since the value of viscosity used here is close to a realistic coronal value, the predictions of viscous heating generated by the anisotropic model give some estimate of the degree of true viscous heating in the solar corona. For the parameters studied, Ohmic heating contributes the bulk of the total heating, outperforming viscous heating by several orders of magnitude. 

As well as enhancing Ohmic heating, the stronger current sheets generated using anisotropic viscosity permit more efficient reconnection, generally enhancing the reconnection rate. This allows for faster magnetic relaxation and, in the case of the disruption of the secondary instability during the nonlinear development of the kink instability in chapter~\ref{chp:kink_instability}, more energetic reconnection-driven flows. Generally, anisotropic viscosity allows greater release of kinetic energy and permits faster flow structures at smaller scales. In the nonlinear development of the kink instability, these faster flows result in the disruption of a secondary instability which is present when isotropic viscosity is used and the flow remains relatively laminar. In simulations where the fluting instability occurs, the mixing caused by the nonlinear development of the instability appears chaotic and over a range of length scales. These results prompt a question for future research; are flows associated with anisotropic viscosity generally turbulent in nature? 

Two phenomena have been observed which were unexpected: the fluting instability in a dynamically twisted flux rope, and the spontaneous collapse of a twisted null point. Both the fluting instability and the specific mechanism causing the collapse of the null point are pressure driven and rely on spatial gradients in Ohmic heating to generate the required pressure gradients. It is unclear if the use of lower, more realistic values for the resistivity would provide the required Ohmic heating to generate these pressure gradients. Both phenomena are affected by the choice of viscosity model but do not rely on either model; the fluting instability is damped by isotropic viscosity in some cases, while the null collapse occurs sooner when anisotropic viscosity is employed. The results presented here are a proof-of-concept, that the fluting instability can be dynamically excited in a coronal loop model, and that a null point can spontaneously collapse under the action of a torsional driver. The investigation of either would be valuable avenues of future research.

The models of anisotropic viscosity employed throughout this thesis have mainly been the switching model with either the von Mises switching function or the parallel Braginskii switching function, both of which neglect the drift and perpendicular terms in Braginskii's full viscous stress tensor on the basis that the relative strengths of the associated transport parameters render the perpendicular and drift terms negligible. However, a finding general to all numerical experiments performed here has been the consistently smaller length scales and faster flows generated when using anisotropic viscosity, resulting in greater velocity shears. Are these shears great enough to result in notable perpendicular viscosity? Although the drift terms do not participate in heating, how might they affect the dynamics in a strong shear layer such as that found to be susceptible to the KHI in chapter~\ref{chp:null_point_khi}? While these are important questions to answer in future research, without extreme velocity shears the perpendicular and drift components of the Braginskii tensor remain secondary in importance to the parallel and isotropic components. Hence, the switching model provides a valuable numerical tool in the numerical simulation of coronal instabilities.

This thesis presents an initial, important foray into the application of anisotropic models of viscosity to nonlinear solar instabilities. Two novel phenomena, the nonlinear fluting instability in a coronal loop and the spontaneous collapse of a null point, have been presented and warrant further investigation. The switching model provides a useful numerical tool to investigate the balance of isotropic and parallel viscosity in this regard but should be further compared to the full Braginskii model to better understand its applicability in applications involving strong velocity shears.

%Chapter~\ref{chp:background} introduces the physics of the solar corona, the governing magnetohydrodynamic equations and Braginskii's model of anisotropic viscosity. In chapter~\ref{chp:numerical_methods} the construction of a 1D hydrodynamics code using the Lagrangian-remap numerical scheme provides an illustration of the methods underpinning the 3D magnetohydrodynamics code Lare3d which is used to perform the numerical experiments in the remainder of the thesis.

%In chapter~\ref{chp:switching_model} four models of anisotropic viscosity and a single isotropic model are applied to a dynamically twisted magnetic null point and the results compared. The anisotropic models are all based on the full Braginskii model which, for coronal purposes, includes parallel and perpendicular terms but neglects the drift terms. This Braginskii tensor with the drift terms neglected provides a useful benchmark to compare against the switching models. The family of switching models are also presented, all of which interpolate between purely isotropic and purely parallel viscosity based on a measure of anisotropy which depends on the local magnetic field strength. The three given interpolation functions (or measures of anisotropy) are the von Mises switching function, a phenomenological model which describes the local degree of anisotropy using a probabilistic approach, and the parallel and isotropic switching functions, each of which co-opts the coefficients of the parallel and isotropic terms of the Braginskii tensor to describe local anisotropy. When the models are applied to the slowly twisted null point, the isotropic model is shown to overestimate the heating output by two orders of magnitude, compared to the physically more realistic anisotropic models, while the anisotropic models produce relatively similar levels of heating. 

%In chapter~\ref{chp:kink_instability} the kink instability is excited in an initially static, twisted magnetic flux rope with zero net current and the differences between the isotropic model and the von Mises switching model investigated. Since there are no null points, the switching model effectively reverts to fully parallel viscosity. The linear and nonlinear development of the kink instability is studied over a wide range of diffusion parameters, varying both viscosity $\nu$ and resistivity $\eta$ between $10^{-5}$ and $10^{-3}$. While the viscosity marginally affects the linear growth of the instability, the differences in the nonlinear development are stark. When the isotropic model is used, the flows are damped and remain relatively laminar, so much so that a secondary instability forms, while the switching model permits faster flows at smaller length scales which disrupts the conditions capable of sustaining the secondary instability. The switching model also permits the formation of smaller, stronger current sheets which quickens the relaxation of the magnetic field, enhances magnetic reconnection and promotes greater Ohmic heating. This behaviour generalises over the investigated parameters. Overall, the viscous heating generated by the isotropic model is overestimated by approximately two orders of magnitude, compared to the switching model.

%Chapter~\ref{chp:kink_instability_straight} extends the investigation of the kink instability by twisting an initially straight flux tube until it becomes unstable to both the fluting and kink instabilities, in contrast to the flux tube investigated in chapter~\ref{chp:kink_instability} which is prescribed as unstable to the kink instability. This is also investigated over a range of diffusion parameters and, for some parameter choices it is found that the tube can be simultaneously unstable to a fluting instability.  As the tube is twisted, current structures form with the peak current occurring at the axis. Through Ohmic heating, this generates an outwardly-directed pressure force which acts in opposition to the inwardly-directed magnetic tension force associated with the twist to generate, in some cases, a fluting instability. In all cases the tube then becomes unstable to the helical kink instability. For some parameter choices the isotropic viscosity damps (but does not completely suppress) the fluting instability such that its development is completely disrupted by the kink instability, while the switching model permits faster fluting growth prior to the onset of the kink. Where the fluting instability occurs, its development appears to slow the growth of the kink by mixing the interior of the flux tube.

%- new fluting instability

%fluting found with the switching model but damped with the iso model, brief discussion of how this comes about

%In chapter~\ref{chp:null_point_khi} the switching and isotropic models are compared when applied to a magnetic null point which is twisted so rapidly that the Kelvin-Helmholtz instability is able to develop. Additionally, the chapter details the collapse of the null due to a pressure-driven asymmetry.

%- new mechanism of null collapse

%again, highlight the differences between the two viscosity model cases

%General findings
%- Isotropic viscosity overestimates viscous heating (except where switching viscosity enhances its own heating through, say, KHI)
%- Switching viscosity enhances Ohmic heating
%- Switching visc leads to smaller length scales, stronger currents, greater vorticity, faster flows
%- Isotropic viscosity can notably damp instabilities
%- Or construct new ones (e.g. kink ins)

%general findings are good, in particular that although parallel viscosity produces less viscous heating than iso visc., its effect is to allow for the development of many more fine-scale current sheets, which in turn leads to more ohmic heating. Thus the viscosity has an indirect but important influence on nonlinear development.


%finish off with something about this study being an initial but comprehensive step to understanding the effect of viscosity on the nonlinear development of instabilities in the corona.
